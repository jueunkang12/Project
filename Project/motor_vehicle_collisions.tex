% Options for packages loaded elsewhere
\PassOptionsToPackage{unicode}{hyperref}
\PassOptionsToPackage{hyphens}{url}
\PassOptionsToPackage{dvipsnames,svgnames,x11names}{xcolor}
%
\documentclass[
  letterpaper,
  DIV=11,
  numbers=noendperiod]{scrartcl}

\usepackage{amsmath,amssymb}
\usepackage{lmodern}
\usepackage{iftex}
\ifPDFTeX
  \usepackage[T1]{fontenc}
  \usepackage[utf8]{inputenc}
  \usepackage{textcomp} % provide euro and other symbols
\else % if luatex or xetex
  \usepackage{unicode-math}
  \defaultfontfeatures{Scale=MatchLowercase}
  \defaultfontfeatures[\rmfamily]{Ligatures=TeX,Scale=1}
\fi
% Use upquote if available, for straight quotes in verbatim environments
\IfFileExists{upquote.sty}{\usepackage{upquote}}{}
\IfFileExists{microtype.sty}{% use microtype if available
  \usepackage[]{microtype}
  \UseMicrotypeSet[protrusion]{basicmath} % disable protrusion for tt fonts
}{}
\makeatletter
\@ifundefined{KOMAClassName}{% if non-KOMA class
  \IfFileExists{parskip.sty}{%
    \usepackage{parskip}
  }{% else
    \setlength{\parindent}{0pt}
    \setlength{\parskip}{6pt plus 2pt minus 1pt}}
}{% if KOMA class
  \KOMAoptions{parskip=half}}
\makeatother
\usepackage{xcolor}
\setlength{\emergencystretch}{3em} % prevent overfull lines
\setcounter{secnumdepth}{-\maxdimen} % remove section numbering
% Make \paragraph and \subparagraph free-standing
\ifx\paragraph\undefined\else
  \let\oldparagraph\paragraph
  \renewcommand{\paragraph}[1]{\oldparagraph{#1}\mbox{}}
\fi
\ifx\subparagraph\undefined\else
  \let\oldsubparagraph\subparagraph
  \renewcommand{\subparagraph}[1]{\oldsubparagraph{#1}\mbox{}}
\fi

\usepackage{color}
\usepackage{fancyvrb}
\newcommand{\VerbBar}{|}
\newcommand{\VERB}{\Verb[commandchars=\\\{\}]}
\DefineVerbatimEnvironment{Highlighting}{Verbatim}{commandchars=\\\{\}}
% Add ',fontsize=\small' for more characters per line
\usepackage{framed}
\definecolor{shadecolor}{RGB}{241,243,245}
\newenvironment{Shaded}{\begin{snugshade}}{\end{snugshade}}
\newcommand{\AlertTok}[1]{\textcolor[rgb]{0.68,0.00,0.00}{#1}}
\newcommand{\AnnotationTok}[1]{\textcolor[rgb]{0.37,0.37,0.37}{#1}}
\newcommand{\AttributeTok}[1]{\textcolor[rgb]{0.40,0.45,0.13}{#1}}
\newcommand{\BaseNTok}[1]{\textcolor[rgb]{0.68,0.00,0.00}{#1}}
\newcommand{\BuiltInTok}[1]{\textcolor[rgb]{0.00,0.23,0.31}{#1}}
\newcommand{\CharTok}[1]{\textcolor[rgb]{0.13,0.47,0.30}{#1}}
\newcommand{\CommentTok}[1]{\textcolor[rgb]{0.37,0.37,0.37}{#1}}
\newcommand{\CommentVarTok}[1]{\textcolor[rgb]{0.37,0.37,0.37}{\textit{#1}}}
\newcommand{\ConstantTok}[1]{\textcolor[rgb]{0.56,0.35,0.01}{#1}}
\newcommand{\ControlFlowTok}[1]{\textcolor[rgb]{0.00,0.23,0.31}{#1}}
\newcommand{\DataTypeTok}[1]{\textcolor[rgb]{0.68,0.00,0.00}{#1}}
\newcommand{\DecValTok}[1]{\textcolor[rgb]{0.68,0.00,0.00}{#1}}
\newcommand{\DocumentationTok}[1]{\textcolor[rgb]{0.37,0.37,0.37}{\textit{#1}}}
\newcommand{\ErrorTok}[1]{\textcolor[rgb]{0.68,0.00,0.00}{#1}}
\newcommand{\ExtensionTok}[1]{\textcolor[rgb]{0.00,0.23,0.31}{#1}}
\newcommand{\FloatTok}[1]{\textcolor[rgb]{0.68,0.00,0.00}{#1}}
\newcommand{\FunctionTok}[1]{\textcolor[rgb]{0.28,0.35,0.67}{#1}}
\newcommand{\ImportTok}[1]{\textcolor[rgb]{0.00,0.46,0.62}{#1}}
\newcommand{\InformationTok}[1]{\textcolor[rgb]{0.37,0.37,0.37}{#1}}
\newcommand{\KeywordTok}[1]{\textcolor[rgb]{0.00,0.23,0.31}{#1}}
\newcommand{\NormalTok}[1]{\textcolor[rgb]{0.00,0.23,0.31}{#1}}
\newcommand{\OperatorTok}[1]{\textcolor[rgb]{0.37,0.37,0.37}{#1}}
\newcommand{\OtherTok}[1]{\textcolor[rgb]{0.00,0.23,0.31}{#1}}
\newcommand{\PreprocessorTok}[1]{\textcolor[rgb]{0.68,0.00,0.00}{#1}}
\newcommand{\RegionMarkerTok}[1]{\textcolor[rgb]{0.00,0.23,0.31}{#1}}
\newcommand{\SpecialCharTok}[1]{\textcolor[rgb]{0.37,0.37,0.37}{#1}}
\newcommand{\SpecialStringTok}[1]{\textcolor[rgb]{0.13,0.47,0.30}{#1}}
\newcommand{\StringTok}[1]{\textcolor[rgb]{0.13,0.47,0.30}{#1}}
\newcommand{\VariableTok}[1]{\textcolor[rgb]{0.07,0.07,0.07}{#1}}
\newcommand{\VerbatimStringTok}[1]{\textcolor[rgb]{0.13,0.47,0.30}{#1}}
\newcommand{\WarningTok}[1]{\textcolor[rgb]{0.37,0.37,0.37}{\textit{#1}}}

\providecommand{\tightlist}{%
  \setlength{\itemsep}{0pt}\setlength{\parskip}{0pt}}\usepackage{longtable,booktabs,array}
\usepackage{calc} % for calculating minipage widths
% Correct order of tables after \paragraph or \subparagraph
\usepackage{etoolbox}
\makeatletter
\patchcmd\longtable{\par}{\if@noskipsec\mbox{}\fi\par}{}{}
\makeatother
% Allow footnotes in longtable head/foot
\IfFileExists{footnotehyper.sty}{\usepackage{footnotehyper}}{\usepackage{footnote}}
\makesavenoteenv{longtable}
\usepackage{graphicx}
\makeatletter
\def\maxwidth{\ifdim\Gin@nat@width>\linewidth\linewidth\else\Gin@nat@width\fi}
\def\maxheight{\ifdim\Gin@nat@height>\textheight\textheight\else\Gin@nat@height\fi}
\makeatother
% Scale images if necessary, so that they will not overflow the page
% margins by default, and it is still possible to overwrite the defaults
% using explicit options in \includegraphics[width, height, ...]{}
\setkeys{Gin}{width=\maxwidth,height=\maxheight,keepaspectratio}
% Set default figure placement to htbp
\makeatletter
\def\fps@figure{htbp}
\makeatother

\KOMAoption{captions}{tableheading}
\makeatletter
\makeatother
\makeatletter
\makeatother
\makeatletter
\@ifpackageloaded{caption}{}{\usepackage{caption}}
\AtBeginDocument{%
\ifdefined\contentsname
  \renewcommand*\contentsname{Table of contents}
\else
  \newcommand\contentsname{Table of contents}
\fi
\ifdefined\listfigurename
  \renewcommand*\listfigurename{List of Figures}
\else
  \newcommand\listfigurename{List of Figures}
\fi
\ifdefined\listtablename
  \renewcommand*\listtablename{List of Tables}
\else
  \newcommand\listtablename{List of Tables}
\fi
\ifdefined\figurename
  \renewcommand*\figurename{Figure}
\else
  \newcommand\figurename{Figure}
\fi
\ifdefined\tablename
  \renewcommand*\tablename{Table}
\else
  \newcommand\tablename{Table}
\fi
}
\@ifpackageloaded{float}{}{\usepackage{float}}
\floatstyle{ruled}
\@ifundefined{c@chapter}{\newfloat{codelisting}{h}{lop}}{\newfloat{codelisting}{h}{lop}[chapter]}
\floatname{codelisting}{Listing}
\newcommand*\listoflistings{\listof{codelisting}{List of Listings}}
\makeatother
\makeatletter
\@ifpackageloaded{caption}{}{\usepackage{caption}}
\@ifpackageloaded{subcaption}{}{\usepackage{subcaption}}
\makeatother
\makeatletter
\@ifpackageloaded{tcolorbox}{}{\usepackage[many]{tcolorbox}}
\makeatother
\makeatletter
\@ifundefined{shadecolor}{\definecolor{shadecolor}{rgb}{.97, .97, .97}}
\makeatother
\makeatletter
\makeatother
\ifLuaTeX
  \usepackage{selnolig}  % disable illegal ligatures
\fi
\IfFileExists{bookmark.sty}{\usepackage{bookmark}}{\usepackage{hyperref}}
\IfFileExists{xurl.sty}{\usepackage{xurl}}{} % add URL line breaks if available
\urlstyle{same} % disable monospaced font for URLs
\hypersetup{
  pdftitle={Motor Vehicle Collisions},
  pdfauthor={Jueun Kang},
  colorlinks=true,
  linkcolor={blue},
  filecolor={Maroon},
  citecolor={Blue},
  urlcolor={Blue},
  pdfcreator={LaTeX via pandoc}}

\title{Motor Vehicle Collisions}
\author{Jueun Kang}
\date{}

\begin{document}
\maketitle
\begin{abstract}
This is my abstract.
\end{abstract}
\ifdefined\Shaded\renewenvironment{Shaded}{\begin{tcolorbox}[boxrule=0pt, interior hidden, sharp corners, enhanced, frame hidden, breakable, borderline west={3pt}{0pt}{shadecolor}]}{\end{tcolorbox}}\fi

\begin{Shaded}
\begin{Highlighting}[]
\DocumentationTok{\#\#\#\# Preamble \#\#\#\#}
\CommentTok{\# Purpose: insert purpose here}
\CommentTok{\# Author: Jueun Kang}
\CommentTok{\# Email: jueun.kang@mail.utoronto.ca}
\CommentTok{\# Date: 29 January 2023}
\CommentTok{\# Prerequisites: Need to know where to get Motor Vehicle Collisions data.}
\end{Highlighting}
\end{Shaded}

\begin{Shaded}
\begin{Highlighting}[]
\FunctionTok{library}\NormalTok{(tidyverse)}
\end{Highlighting}
\end{Shaded}

\begin{verbatim}
-- Attaching packages --------------------------------------- tidyverse 1.3.2 --
v ggplot2 3.4.0      v purrr   1.0.1 
v tibble  3.1.8      v dplyr   1.0.10
v tidyr   1.2.1      v stringr 1.5.0 
v readr   2.1.3      v forcats 0.5.2 
-- Conflicts ------------------------------------------ tidyverse_conflicts() --
x dplyr::filter() masks stats::filter()
x dplyr::lag()    masks stats::lag()
\end{verbatim}

\begin{Shaded}
\begin{Highlighting}[]
\FunctionTok{library}\NormalTok{(janitor)}
\end{Highlighting}
\end{Shaded}

\begin{verbatim}

Attaching package: 'janitor'

The following objects are masked from 'package:stats':

    chisq.test, fisher.test
\end{verbatim}

\begin{Shaded}
\begin{Highlighting}[]
\FunctionTok{library}\NormalTok{(opendatatoronto)}
\FunctionTok{library}\NormalTok{(lubridate)}
\end{Highlighting}
\end{Shaded}

\begin{verbatim}

Attaching package: 'lubridate'

The following objects are masked from 'package:base':

    date, intersect, setdiff, union
\end{verbatim}

\begin{Shaded}
\begin{Highlighting}[]
\FunctionTok{library}\NormalTok{(opendatatoronto)}
\FunctionTok{library}\NormalTok{(dplyr)}

\CommentTok{\# get package}
\NormalTok{package }\OtherTok{\textless{}{-}} \FunctionTok{show\_package}\NormalTok{(}\StringTok{"0b6d3a00{-}7de1{-}440b{-}b47c{-}7252fd13929f"}\NormalTok{)}
\NormalTok{package}
\end{Highlighting}
\end{Shaded}

\begin{verbatim}
# A tibble: 1 x 11
  title     id    topics civic~1 publi~2 excerpt datas~3 num_r~4 formats refre~5
  <chr>     <chr> <chr>  <chr>   <chr>   <chr>   <chr>     <int> <chr>   <chr>  
1 Motor Ve~ 0b6d~ Publi~ <NA>    Toront~ This d~ Map           9 SHP,CS~ Annual~
# ... with 1 more variable: last_refreshed <date>, and abbreviated variable
#   names 1: civic_issues, 2: publisher, 3: dataset_category, 4: num_resources,
#   5: refresh_rate
\end{verbatim}

\begin{Shaded}
\begin{Highlighting}[]
\CommentTok{\# get all resources for this package}
\NormalTok{resources }\OtherTok{\textless{}{-}} \FunctionTok{list\_package\_resources}\NormalTok{(}\StringTok{"0b6d3a00{-}7de1{-}440b{-}b47c{-}7252fd13929f"}\NormalTok{)}

\CommentTok{\# identify datastore resources; by default, Toronto Open Data sets datastore resource format to CSV for non{-}geospatial and GeoJSON for geospatial resources}
\NormalTok{datastore\_resources }\OtherTok{\textless{}{-}} \FunctionTok{filter}\NormalTok{(resources, }\FunctionTok{tolower}\NormalTok{(format) }\SpecialCharTok{\%in\%} \FunctionTok{c}\NormalTok{(}\StringTok{\textquotesingle{}csv\textquotesingle{}}\NormalTok{, }\StringTok{\textquotesingle{}geojson\textquotesingle{}}\NormalTok{))}

\CommentTok{\# load the first datastore resource as a sample}
\NormalTok{data }\OtherTok{\textless{}{-}} \FunctionTok{filter}\NormalTok{(datastore\_resources, }\FunctionTok{row\_number}\NormalTok{()}\SpecialCharTok{==}\DecValTok{1}\NormalTok{) }\SpecialCharTok{\%\textgreater{}\%} \FunctionTok{get\_resource}\NormalTok{()}
\NormalTok{data}
\end{Highlighting}
\end{Shaded}

\begin{verbatim}
Simple feature collection with 17488 features and 52 fields
Geometry type: POINT
Dimension:     XY
Bounding box:  xmin: -79.63839 ymin: 43.58968 xmax: -79.12297 ymax: 43.85544
Geodetic CRS:  WGS 84
# A tibble: 17,488 x 53
   `_id` ACCNUM  YEAR DATE  TIME  STREET1 STREET2 OFFSET ROAD_~1 DISTR~2 WARDNUM
   <int> <chr>  <int> <chr> <chr> <chr>   <chr>   <chr>  <chr>   <chr>     <int>
 1     1 893184  2006 2006~ 236   WOODBI~ O CONN~ None   Major ~ Toront~      19
 2     2 893184  2006 2006~ 236   WOODBI~ O CONN~ None   Major ~ Toront~      19
 3     3 913296  2006 2006~ 2040  29 BLA~ None    None   Local   Etobic~       2
 4     4 893184  2006 2006~ 236   WOODBI~ O CONN~ None   Major ~ Toront~      19
 5     5 893184  2006 2006~ 236   WOODBI~ O CONN~ None   Major ~ Toront~      19
 6     6 911336  2006 2006~ 1000  BIRCHM~ MODERN~ None   Major ~ Scarbo~      21
 7     7 893184  2006 2006~ 236   WOODBI~ O CONN~ None   Major ~ Toront~      19
 8     8 911336  2006 2006~ 1000  BIRCHM~ MODERN~ None   Major ~ Scarbo~      21
 9     9 893184  2006 2006~ 236   WOODBI~ O CONN~ None   Major ~ Toront~      19
10    10 911336  2006 2006~ 1000  BIRCHM~ MODERN~ None   Major ~ Scarbo~      21
# ... with 17,478 more rows, 42 more variables: LOCCOORD <chr>, ACCLOC <chr>,
#   TRAFFCTL <chr>, VISIBILITY <chr>, LIGHT <chr>, RDSFCOND <chr>,
#   ACCLASS <chr>, IMPACTYPE <chr>, INVTYPE <chr>, INVAGE <chr>, INJURY <chr>,
#   FATAL_NO <chr>, INITDIR <chr>, VEHTYPE <chr>, MANOEUVER <chr>,
#   DRIVACT <chr>, DRIVCOND <chr>, PEDTYPE <chr>, PEDACT <chr>, PEDCOND <chr>,
#   CYCLISTYPE <chr>, CYCACT <chr>, CYCCOND <chr>, PEDESTRIAN <chr>,
#   CYCLIST <chr>, AUTOMOBILE <chr>, MOTORCYCLE <chr>, TRUCK <chr>, ...
\end{verbatim}

\begin{Shaded}
\begin{Highlighting}[]
\CommentTok{\# cleaning the data}
\NormalTok{clean\_data }\OtherTok{\textless{}{-}}
  \FunctionTok{clean\_names}\NormalTok{(data) }\SpecialCharTok{|\textgreater{}}
  \FunctionTok{select}\NormalTok{(}
\NormalTok{    year, }
\NormalTok{    alcohol}
\NormalTok{    ) }\SpecialCharTok{|\textgreater{}}
  \FunctionTok{filter}\NormalTok{(alcohol }\SpecialCharTok{==} \StringTok{"Yes"}\NormalTok{) }\SpecialCharTok{|\textgreater{}}
  \FunctionTok{group\_by}\NormalTok{(year) }\SpecialCharTok{|\textgreater{}}
  \FunctionTok{count}\NormalTok{(alcohol)}

\NormalTok{x }\OtherTok{=} \FunctionTok{count}\NormalTok{(clean\_data, year)}
\end{Highlighting}
\end{Shaded}




\end{document}
